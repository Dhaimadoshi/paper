%%%---%%%---%%%---%%%---%%%---%%%---%%%---%%%---%%%---%%%---%%%---%%%---%%%
\chapter{Introduction}

%%%---%%%---%%%---%%%---%%%---%%%---%%%---%%%---%%%---%%%---%%%---%%%---%%%
This paper will present a \textit{debugging tool} developped to help troubleshooting the design of Autumn grammars. First of all, I'd like to stress the fact that we are discussing a debugging tool as opposed to a \textit{``debugger''} strictly speaking. The reader might expect a debugger to be able to follow the flow of excecution in real time and to be able to stop it at will. Instead, our tool explore another approach: it first let the execution of the parse run completely, gathering all the relevant information needed to reason about the grammar down the road. In that quality, it is more like an code anylizer rather than a debugger stricly speaking.

\bigskip

Notherless, the object of this tool is to help debug a grammar, therefore, we will simply refer to our debugging tool with the word \textit{``debugger''} for the rest of this paper. Please take note of this nomenclature specificity. 

\bigskip

\section{Motivation}

During the lifetime of any project, the chunk dedicated to debugging is usually important and represent a crucial step in the developpement of a solution. Traditionnally, debuggers have given the developpers access to the running systems and general mechanism that expose the raw system state. Although this solution gives a universal framework to debug any project, the fact is developpers think and reason in term of high level abstractions while the debuggers only expose raw data which force the developpers to mentally refine their high level questions into low level ones making the debugging session an unnecessarily difficult and error prone endeavor which in term can increase the developpement duration and cost \cite{economicdebug}.

\bigskip

When applied to our domain of parsing and language recognition, this problem becomes all the more obvious. For example, consider the parse of an input, one might be interested to analyze the behavior of the parser past a certain position in the input stream. Since general purpose debuggers have no knowledge about parsing or input streams, the developper would have to manually single step through the entire execution until we reach the affordmentionned input position.

Moreover, writting grammars is a tricky and error probed business. Errors could come from a mistake in the definition of the grammar or from a mistake in the input stream or even from a mistake in some user defined parser. The highly recursive nature of grammars makes it so that errors reported by the general purpose debugger are rarely useful. Indeed, the debugger having no knowledge about those higher level abstractions cannot inform the developpers with clear information to reason with.

% Debuggers are comprehension tools. direct access to the running system Despite their importance, most debuggers only provide low-level operations that do not capture user intent and standard user interfaces that only display generic information. traditional debuggers =  generic mechanisms to explore and exhibit the execution stack and system state, expose the underlying execution and helps detect issues and correct them

\bigskip

Other debugging solution has been developped for other parsing library, what makes our solution different is the simplicity of our approach combined with the power of Autumn's parsing library to express true context sensitivity.

\bigskip

Nowadays, most modern programming languages exhibit context sensitivity features\footnote{TODO: add ref}, for example, python exhibits context sensitivity through significant whitespace. However, true context-sensitivity has rarely handled by parsing libraries. The reason lies with the difficulties to express context sensitivity with current grammar formalisms. As a result most solutions lack context transparancy (Grammatical construct needs to be unaware of the context shared between its ancestors and its descendants) and instead intertwin context and grammar making it difficult to maintain or eaven reason about.

\bigskip

In the other hand, Autumn is a context sensitive parser combinator library that approach the issue of context transparancy by intoducing the \textbf{``principled stateful parsing''} discipline. It is based on a \textit{stateful parsing} approach in which the context is passed implicitely through a parse wide mutable state. This approach introduce its own set of problems that Autumn solve by introducing a set a primitive operations designed to formally interact with the parse state.

\bigskip
% So the approach is to simply the life of the use by writting a debugger. 


We propose a code inspection tool for Autumn's grammar designed as an Intellij plugin user interface that allow the exposure of higher level concepts relevant to parsing theory such as syntax tree inspector in all simplicity. With this tool, the developper can test grammar rules and track the invocation sequence of the parsers helping him to detect errors accross the project.

To convince ourselves, I will refer to section \ref{debugexample} in which we present several case studies highlighting the possibilities of our solution in comparison to general purpose debuggers.

\bigskip

Our goal is to simplify the life of Autumn grammar developpers. Just as developers use IDEs to dramatically improve their productivity, programmers need a sophisticated development environment for building, understanding, and debugging grammars. 

	\begin{comment}


Despite their importance, most debuggers only provide low-level operations that do not capture user intent and standard user interfaces that only display generic information. These issues can be addressed if developers are able to create domain-specific debuggers adapted to their problems and domains. Domain- specific debuggers can provide features at a higher level of abstraction that (i) match the domain model of software applications and (ii) group contextual information from various sources.

Consider a framework for synchronous message passing. One common use case in applications using it is the delivery of a message to a list of subscribers. When debugging this use case, a developer might need to step to when the current message is delivered to the next subscriber. One solution is to manually
step through the execution until the desired code location is reached. Another consists in identifying the code location beforehand, setting a breakpoint there and resuming execution. In both cases developers have to manually perform a series of actions each time they want to execute this high-level operation.

The lack of a debugger at the proper abstraction level limits an end-user‟s ability to discover and locate faults in a DSL program.

	\end{comment}
%%%---%%%---%%%---%%%---%%%---%%%---%%%---%%%---%%%---%%%---%%%---%%%---%%%---%%%---%%%---%%%---%%%---%%%---%%%---%%%---%%%---%%%---%%%---%%%---%%%

%%%---%%%---%%%---%%%---%%%---%%%---%%%---%%%---%%%---%%%---%%%---%%%---%%%---%%%---%%%---%%%---%%%---%%%---%%%---%%%---%%%---%%%---%%%---%%%---%%%	


	 %The Dump file is really not managable. 