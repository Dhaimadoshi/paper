%%%---%%%---%%%---%%%---%%%---%%%---%%%---%%%---%%%---%%%---%%%---%%%---%%%---%%%---%%%---%%%---%%%---%%%---%%%---%%%---%%%---%%%---%%%---%%%---%%%
%
\chapter{Future work}
%
%%%---%%%---%%%---%%%---%%%---%%%---%%%---%%%---%%%---%%%---%%%---%%%---%%%---%%%---%%%---%%%---%%%---%%%---%%%---%%%---%%%---%%%---%%%---%%%---%%%
	
	%%%---%%%---%%%---%%%---%%%---%%%---%%%---%%%---%%%---%%%---%%%---%%%---%%%---%%%---%%%---%%%---%%%---%%%---%%%---%%%---%%%---%%%---%%%---%%%---%%%
	\section{GUI extentions}
	%%%---%%%---%%%---%%%---%%%---%%%---%%%---%%%---%%%---%%%---%%%---%%%---%%%---%%%---%%%---%%%---%%%---%%%---%%%---%%%---%%%---%%%---%%%---%%%---%%%
	Making use of tornadoFX powerful verboseless feature, we implemented the plugin strictly following the MVC paradigm, decoupling the functionalities from the display. We worked with easy maintainability and extendability in mind at all time. 

	The code is divided in 3 different classes, there is the Model which stores all the data. Then there is the views that define the way data will be displayed. And finally, there is a controller, the middleman that request information to the model and dispatch them to the views. The implementation make use of an event system which help decouple the controller and the view furthermore. It is therefore very easy to create new views to hook on the masterview, or replace the masterview all together, the only thing needed is to listen to the event and display the received data in the chosen way. Symmetrically, it is as easy to create new filters, or provide new information to the views by creating new events and new methods in the controller.

	Because Autumn implements stateful parsing, the users may define custom states for his parsers. It is then his responsability to create a display for it and to feed it to the plugin views. For this purpose I added an interface for the states that define a ``getRepresentation'' method.

	%%%---%%%---%%%---%%%---%%%---%%%---%%%---%%%---%%%---%%%---%%%---%%%---%%%---%%%---%%%---%%%---%%%---%%%---%%%---%%%---%%%---%%%---%%%---%%%---%%%
	\section{Debugger extension}
	%%%---%%%---%%%---%%%---%%%---%%%---%%%---%%%---%%%---%%%---%%%---%%%---%%%---%%%---%%%---%%%---%%%---%%%---%%%---%%%---%%%---%%%---%%%---%%%---%%%

	\subsection{Adding hooks}
	\begin{itemize}
		\item access more informations during debug
		\item create new state structure
	\end{itemize}

	%%%---%%%---%%%---%%%---%%%---%%%---%%%---%%%---%%%---%%%---%%%---%%%---%%%---%%%---%%%---%%%---%%%---%%%---%%%---%%%---%%%---%%%---%%%---%%%---%%%
	\section{Autumn related}
	%%%---%%%---%%%---%%%---%%%---%%%---%%%---%%%---%%%---%%%---%%%---%%%---%%%---%%%---%%%---%%%---%%%---%%%---%%%---%%%---%%%---%%%---%%%---%%%---%%%
		\subsection{Parser implementation}
		\begin{itemize}
			\item as discussed, parsers as been reimplemented as objects.
			\item the implementation was meant to be temporary because of the suspected performances issues
			\item since this issues are neglictable, the functions can be directly integreted in the object implementation
		\end{itemize}


%%%---%%%---%%%---%%%---%%%---%%%---%%%---%%%---%%%---%%%---%%%---%%%---%%%---%%%---%%%---%%%---%%%---%%%---%%%---%%%---%%%---%%%---%%%---%%%---%%%
%
\chapter{validation}
%
%%%---%%%---%%%---%%%---%%%---%%%---%%%---%%%---%%%---%%%---%%%---%%%---%%%---%%%---%%%---%%%---%%%---%%%---%%%---%%%---%%%---%%%---%%%---%%%---%%%

\section{Benchmark}
Benchmark with consequent java code. Because of megamorphic call sites : expected the new implementation of the parsers to be much slower. in practice its not the case.

\section{Debugging Benchmark}
Avec l'ajout de la logique de debug, voila les perfs:

\begin{itemize}
	\item vitesse d'execution comparée avec et sans le debug
	\item memoire consomée par les structures de debug. (discuté si c'est un gros desavantage ou non, si oui apporter une solution pour travailler avec de gros fichier input.)
\end{itemize}



	%%%---%%%---%%%---%%%---%%%---%%%---%%%---%%%---%%%---%%%---%%%---%%%---%%%---%%%---%%%---%%%---%%%---%%%---%%%---%%%---%%%---%%%---%%%---%%%---%%%
	\section{Debugger in practice}
	%%%---%%%---%%%---%%%---%%%---%%%---%%%---%%%---%%%---%%%---%%%---%%%---%%%---%%%---%%%---%%%---%%%---%%%---%%%---%%%---%%%---%%%---%%%---%%%---%%%

	\subsection{Examples}

	\begin{itemize}
		\item case study 1 : error in grammar
		\item case study 2 : error in input
		\item case study 3 : error in parser definition
		\item show how the debugger helps detecting errors in the differrent case studies
	\end{itemize}