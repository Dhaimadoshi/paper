%%%---%%%---%%%---%%%---%%%---%%%---%%%---%%%---%%%---%%%---%%%---%%%---%%%---%%%---%%%---%%%---%%%---%%%---%%%---%%%---%%%---%%%---%%%---%%%---%%%
%
\chapter{Overview of the solution}
%
%%%---%%%---%%%---%%%---%%%---%%%---%%%---%%%---%%%---%%%---%%%---%%%---%%%---%%%---%%%---%%%---%%%---%%%---%%%---%%%---%%%---%%%---%%%---%%%---%%%
This chapter presents the main functionality of the debugging tool in all generalities and presents how it has been integrated with Intellij IDE as a GUI plugin. At this point I'd like to remind the reader that our solution is not a \textit{debuger} in the traditional meaning of the word. Dispite this fact, we will refer to it as a debuger for simplicity. 

%%%---%%%---%%%---%%%---%%%---%%%---%%%---%%%---%%%---%%%---%%%---%%%---%%%---%%%---%%%---%%%---%%%---%%%---%%%---%%%---%%%---%%%---%%%---%%%---%%%
\section{Overview}
%%%---%%%---%%%---%%%---%%%---%%%---%%%---%%%---%%%---%%%---%%%---%%%---%%%---%%%---%%%---%%%---%%%---%%%---%%%---%%%---%%%---%%%---%%%---%%%---%%%
To empower Autumn's parsing library with a proper debugging tool, we developped an IDE plugin designed to be a straple for the developper to interact with the grammar's output and track down errors. During a debugging session, Autumn will attempt to parse the entire input. Weather it is successful or not, it produces a parse tree during its invocation.

\bigskip

The generated parse tree is then passed to the plugin's GUI. Each node of the tree represents the invocation of a single parser and contains information about the circumstances surrounding its invocation that can be used to assess the correctness of its behavior

\bigskip

The tree itself can be seen as a list representing the history of invocation, the first entry being the starting expression. Each entries representing a moment in time during the parse, one can retrace the execution of the entire parse, stepping forward or backward simply by navigating up and down the list. 

\bigskip

The debugging effort can therefore be summurized as a search in a tree of event.


%%%---%%%---%%%---%%%---%%%---%%%---%%%---%%%---%%%---%%%---%%%---%%%---%%%---%%%---%%%---%%%---%%%---%%%---%%%---%%%---%%%---%%%---%%%---%%%---%%%
\subsection{Why not a ``normal'' debuger ?}
%%%---%%%---%%%---%%%---%%%---%%%---%%%---%%%---%%%---%%%---%%%---%%%---%%%---%%%---%%%---%%%---%%%---%%%---%%%---%%%---%%%---%%%---%%%---%%%---%%%

From the start, we wanted to create a plugin that would serve as a GUI for our debugger. The original idea was to create a more conventional debuger by overlaying our logic on top of Intellij's general purpose debugger to do achieve a more traditional approach allowing us to step live into the execution. The problem is that Intellij doesn't expose the implementation of its debugger through the plugin interface, making it a much more difficult endeavor. To access the IDE's debugger, one could work directly with the community version of the IDE which is open source, but we decided to stick with our plugin interface.

\bigskip

Facing this problem made us think the solution over in term of practical use. We tried to put ourselves in the shoes of grammar developpers and asked ourselves questions like: 

\begin{itemize}
	\item What problem will I face while developping a grammar ?
	\item What information do I need to track errors down and take decisions ?
	\item How can I get those informations in the simplest possible way ?
\end{itemize}

\subsection{Difficulties of grammar development} 

The most common problem encountered by grammar developpers is to determine why a generated parser incorrectly interprets an input sentence. Generally, an incorrect parse can be reduced to three different cases:

\begin{itemize}
	\item The grammar contains a certain number of wrong production rules that leads to a wrong interpretation of the input.
	\item The input itself contains incorrect parts with respect to the specification of the grammar which in turns lead to a wrong interpretation of the input.
	\item There is a mistake in the definition of some user custom parser.
\end{itemize}

\paragraph{TODO} Having some examples of bad error reporting and highliting the difficulties of grammar debugging here might be a good place

\bigskip

To deal with those issues, the most important properties that we need are :

\begin{itemize}
	\item to able to expose chronologically call sequence of each parser.% This is traditionally done by stepping into the next instruction in traditional debugging. 
	\item the ability to manipulate the input stream and identify what portion of the input a particular parser has matched. Additionally to be able to analyze the parsing behavior at a certain position in the input is also very valuable. %Traditional parsers have no knowledge of input stream manipulation, therefore the only way to accomplish this is to manually step into the code instruction by instruction which is time consumming and inconvenient.
	\item lastly, to be able to inspect the condition of the general parse state during the call of a particular parser. %Similarly to the previous point, traditional parsers have no knowldege about parse state.
\end{itemize}

As we discussed before, general purposed debugger fall short of dealing with those particular points. Indeed, general purpose debuggers don't have any knowledge of input stream manipulation or parse state, therefore the only way to gather those information using a general purposed debugger is to manually step into the code and monitor mentally the mutation of some low level data structures which is far from convenient.

\bigskip

We wanted to be able to observe the flow of execution of the parse at a higher level of abstraction than instruction by instruction. We wanted to be able to travel through time at the parser level, being able to see which parser was called in which order, being able to see where a parse failed and to be able to go back in time to analyze if the condition for it to succeed was theorically met or if the problem came from elsewhere. We wanted to be able to jump immediately to a certain point in the input stream and see how the parser matched a particular portion of it. And finally, we wanted to be able to analyze the parse state at any point during the execution of any parser.

\bigskip

The question was then, instead of enhancing the general debuger with the required abstractions, could we do that in another way ? Our alternate approach proved to be much simpler indeed. By building a parse tree that can be navigated, we meet the first requirement of analyzing the flow or sequence of invocation of the parsers as well as exposing the hierachy of the parsers.

Autumn's formalism can be specified as a ``functional flow'' where one can plug an arbitrary function that consults the input and push the parse forward or fail. The only information required to simulate input stream manipulation is the position in the input that the parser consulted. It is not unreasonable to reccord a single integer for each node of the tree, that leaves the lasst problem of being able to analyze the parse state at any point.

\bigskip

When manipulating the parse state, autumn keep a log of every mutation a parser applied to it. This structure is maintained primarely to revert the mutation a backtracking parser applied to the state. This structure serves our purpose as well, all we have to do is to record the size of the log for each parser. From this information, it is easy to regenerate the parse state as it were during the invocation of a specific parser.

\bigskip

Our alternate solution to traditional debugging is build on this idea, building a parse tree and recording the input position as well as the log size for each parser. Navigating this tree allow us to observe the flow of execution just as we were travelling back and forth through time at the parser's invocation. 
	
	% the syntax tree generated by the parse can by analyze and we can simulate time traveling in the parse execution
%%%---%%%---%%%---%%%---%%%---%%%---%%%---%%%---%%%---%%%---%%%---%%%---%%%---%%%---%%%---%%%---%%%---%%%---%%%---%%%---%%%---%%%---%%%---%%%---%%%
\section{Functionality}
%%%---%%%---%%%---%%%---%%%---%%%---%%%---%%%---%%%---%%%---%%%---%%%---%%%---%%%---%%%---%%%---%%%---%%%---%%%---%%%---%%%---%%%---%%%---%%%---%%%


	\paragraph{Syntax tree}
	Originally, autumn doesn't build a syntax tree like it is generally done in other parsing libraries. Instead it directly generate a abstract syntaxt tree (AST) which contract nodes together to make them easier to read from a human stand-point. Such an AST doesn't represent the grammar as closely as a regular syntax tree as we no longer have a one to one relationship between the grammar rules and nodes of the tree.

	\bigskip

	Having a one to one relationship between the rules of the grammar and the node of the tree is a crutial property for our debugging purpose. Indeed, we navigate through the tree to observe the invocation of the parsers through time, so if a particular node is actually the abstraction of several children parser, it can be that one node doesn't correspond to one rule of the grammar but to several. To be able to tell which one contains errors would be made much more difficult and less relevant.

	\bigskip

	The debugger assumes the task of building the required syntax tree. Since both trees contain essentially the same information, we could easily have had the Autumn's logic build the syntax tree at the same time it does the AST to improve the performances and avoid uneccessary work. Nonetheless, we choose to seperate the debugging logic from autumn's implementation as best we could to minimize coupling between the debugger and Autumn. Autumn's logic doesn't need to build the syntax tree and so this effort should be undertaken only in the debugging context.

	\begin{comment}
		\item Autumn implementation of parser as function -> changed into objects to create hooks for debugger.
		\item neet to rewrite java grammar to reflect changes in the parsers implementation

		\item each node of the syntax tree contains informations about the parse state at the moment of the parser invocation
		\item those information can then be processed and displayed on a GUI for the developper to reason about
		\item ULM chart of the debugger and how it connects to autumn
	\end{comment}



	% We developped a debugger, to do so we had to adapt autumn parsers. Parsers were initially boolean functions, we transformed them into objects that can be manipulated and stored to remember which parsers were called and in which order during runtime
	\paragraph{Gathering relevant information}
	The syntax tree is build implicitely during the invocation of each parser. Every time a parser is called, a new node is create in the tree and some information about the context is recorded within the node. The information reccorded in the node are:
	\begin{itemize}
		\item the \textbf{name of the production rule} the parser is defining, if any. Each production rule being defined by a combination of parsers, some parser are just a part of the definition of such rule.
		\item the \textbf{type of the parser}. (i.e. Seq, Str, \dots)
		\item its \textit{parent} and \textit{children}
		\item and the \textbf{log size} to regenerate the parse state
	\end{itemize}

%%%---%%%---%%%---%%%---%%%---%%%---%%%---%%%---%%%---%%%---%%%---%%%---%%%---%%%---%%%---%%%---%%%---%%%---%%%---%%%---%%%---%%%---%%%---%%%---%%%
\section{Intellij plugin}
%%%---%%%---%%%---%%%---%%%---%%%---%%%---%%%---%%%---%%%---%%%---%%%---%%%---%%%---%%%---%%%---%%%---%%%---%%%---%%%---%%%---%%%---%%%---%%%---%%%

	% All these information can then be displayed on a GUI.

	The plugin we developped is essentially composed of a panel that can be docked on any side of the IDE window, although it is recommanded to dock in at the bottom or on floating on another screen for maximum clarity. Figure \ref{fig:intj_plug} shows how the plugin integrate within the IDE.

	\begin{figure}[h]
		\includegraphics[width=1\textwidth] {ressources/intj_plug}
		\caption{IDE GUI for Autumn's debuggin tool} 
		\label{fig:intj_plug}
	\end{figure}

	The parse tree that has been build can be displayed in two different forms. A table form and a tree form, the former represents the execution trace, it displays each parser in the sequence they were called in a simple list. The later displays the same information as a tree, highlighting the hierachy between parsers.

	\bigskip

	It can be argued that the tree view would be enough and that having a table view would be redundant. While this is true, given the highly recursive nature of grammar rules, navigating the tree can quickly lead to very high level of nesting, while it makes the hierachy between parsers obvious it also makes it more difficult to have a clear image of the sequence of calls. On the other hand, the table view being a simple list makes it really easy to follow the sequence of call one by one, the downside being that it hides the parser hierachy. So even if it can argued that both views are redundant and unecessary, we think that in some cases, it can provide a small improvement in the quality of life of the developper which is what this tool is all about.

	\bigskip

	It can also be argued that this information as is wouldn't be very useful to the developper as it is quiet a dump of information. When trying to debug a grammar, most of the time we are interested in the behavior of a restricted number of rules that we suspect are carrier of mistakes. In traditional debugging, this is done by setting a serie of breakpoints and executing the code. Once the breakpoint occurs, the programmer will generally single step through the code instructions by instructions. %In this domain, it is rarely useful to single step through each instructions as you are more interested in the behavior of the rules as a whole rather than the sum of each of its parts.
	We achieve this by introcing a serie of filters and search field. By filtering the information, one can really easily isolate the execution of a parser or rule he is interested to verify the correctness and behavior. From there he can easily navigate the tree up and down to see how the grammar parsed the input around those particular parsers and reason about them.

	\bigskip

	Note that the plugin can also run as a stand alone application as shown on figure \ref{fig:stand_alone}, albeit some disabled functionnalities (those that are tightly coupled to the IDE, like jumping to the rule definition). The usefulness of such feature can be discussed, it could possibly serve as a starting point to develop a full fledged independent debugging environment for autumn grammars. 

	\begin{figure}[H]
		\centering
		\includegraphics[width=.75\textwidth] {ressources/stand_alone}
		\caption{Stand alone view - the debugger GUI can be executed independetly from the IDE albeit some restrictions} 
		\label{fig:stand_alone}
	\end{figure}

%%%---%%%---%%%---%%%---%%%---%%%---%%%---%%%---%%%---%%%---%%%---%%%---%%%---%%%---%%%---%%%---%%%---%%%---%%%---%%%---%%%---%%%---%%%---%%%---%%%
	\subsection{Views}
%%%---%%%---%%%---%%%---%%%---%%%---%%%---%%%---%%%---%%%---%%%---%%%---%%%---%%%---%%%---%%%---%%%---%%%---%%%---%%%---%%%---%%%---%%%---%%%---%%%
	This section describe the content of the different views in a little more details.

	\bigskip

	\paragraph{Figure \ref{fig:stand_alone_detailed} presents the different elements of the GUI.}
	\begin{itemize}
		\item 1. Debug button, used to start the debugging process
		\item 2. Filters buttons, used to apply filters to the displayed data
		\item 3. Search field, used to filter the parsers by name
		\item 4. Main information display, 3 tabs are available : Table view, Tree view and finally state view
		\item 5. Side tab panel, used to switch from the main view to the rule view.
	\end{itemize}

	\begin{figure}[t]
		\includegraphics[width=1\textwidth] {ressources/stand_alone_detailed}
		\caption{Close up view of the plugin in detail} 
		\label{fig:stand_alone_detailed}
	\end{figure}

	\subsubsection{Table view - Execution trace} An example of the table view displaying information of a small parse example can be seen on figure \ref{fig:tableview}. As discussed before, the table representation of information allow for clearer display of the sequence of parser calls, it countains the parser calls in chronological order.

	\bigskip

	The table is divided in a serie column:
	\begin{itemize}
		\item Column rule lists the name of the production rule that the parse is defining
		\item Column type lists the actual parser type
		\item log size represents as its name indicate the size of the log, the number of mutation applied to the parse state so far.
		\item pos0 represents the position in the input before the parser was invoked while pos indicate the position of the input after the parser was executed.
		\item finally, backtracked indicates with a boolean property if the parser was successful or if it backtracked. 
	\end{itemize}


	% The table view highlight the execution sequence of the parsers. It is a list  This view can be considered redundant with the tree view which offer the same information but displayed in a tree fashion. Dispite it's redundancy, it shouldn't be dismissed too eagerly, in a tree view, the potentially high nesting of each nodes can create situation where two chronological consecutive parser calls happen in two very different nesting level making the structure harder to read when one focus on the chronological property of events. This is the reason why we considered this view to be important as well, 

	\bigskip

	\begin{figure}[h]
		\centering
		\includegraphics[width=.7\textwidth] {ressources/tableview}
		\caption{Table view: Chronological parser call on a trivial parse example} 
		\label{fig:tableview}
	\end{figure}

	\paragraph{Context menu} In this view there is a single context menu called ``show rule info''. When clicked, it displays the complete definition of the grammar rule as well as highlighting the portion of the input that was matched during the parse.  


	\subsubsection{Tree view - Syntax tree} The tree view displays the data as a tree like structure. It highlights the hierarchy between parsers making it obviously easy to trace where the invocation call came from. When first generated, the structure opens two level of nesting by default. The children of a particular node can be displayed by double clicking on it, as a side effects, the column gets resized to accomodate the new nesting level. An example of this view can be seen on figure \ref{fig:treeview} for a simple parse example.
	
	\bigskip

	\begin{figure}[h]
		\centering
		\includegraphics[width=.7\textwidth] {ressources/treeview}
		\caption{} 
		\label{fig:treeview}
	\end{figure}

	The context menu for this view is a little bit more complex and contains four different actions.
	\paragraph{Context menu - Set as root} Because of the reccursive nature of the parse tree, high level of nesting can quickly disrupt the clarity of the display. This option does exactly what it suggest it does, it sets the selected rule as the root of the tree, displaying only its children. It is useful when analyzing the parse of a particular rule deep inside the tree to isolate it from the rest of its siblings. The resulting view is a subset of the original tree, improving the clarity of the display by focusing on the elements one is reasoning about.

	\paragraph{Context menu - Set parent as root} This option works similarily to the previous one, it take the parent of the root and sets it as the new root. This is useful when one is tracking an error, one analyzes the parse of one rule, and after making sure of its correctness, one might go one step back and analyze the parse of it's parent.

	\paragraph{Context menu - Reset root} Finally, at any point, one might want to display the entire tree again. This option reset the root to the starting expression.

	\paragraph{Context menu - Analyze entry} The last context menu option is used to display the definition of the rule and highlight the matched input. 

	\subsubsection{State information} The last tab of the main view's purpose is to display the parse state associated to the parser currently inspected. In the current implementation of Autumn, the parse state is simply used to record the position of the input and the abstract syntax tree built during the parse. Both information are already displayed in the other views, therefore, it doesn't display information for now. This tab remains relevant notherless because users can implement their own structures to store custom parse state providing they respect Autumn's prescription for such structure. Those custom structure could contain relevant information that has not been captured by the other views. The existance of this tab, and a hook that has been implemented in those structure can greatly facilitate the display of such custom information.

	\subsubsection{Rule side tab} 

	The last section of the plugin we didn't discuss yet is the side tab displays. When one is analyzing a parser using the context menu of the other views, this panel's information is populated with the rule definition of the parser and the portion of input matched by it. Figure \ref{fig:ruledis} shows an example of that.
	The top part of the panel, above the line, is the definition of the production rule linked to the parser. Below the line is the portion of the input matched by the parser.

	\begin{figure}[h]
		\includegraphics[width=1\textwidth] {ressources/ruledis}
		\caption{Example of grammar rule information and matched input for a particular parser} 
		\label{fig:ruledis}
	\end{figure}

	%%%---%%%---%%%---%%%---%%%---%%%---%%%---%%%---%%%---%%%---%%%---%%%---%%%---%%%---%%%---%%%---%%%---%%%---%%%---%%%---%%%---%%%---%%%---%%%---%%%
	\subsection{Filtering the informations}
	%%%---%%%---%%%---%%%---%%%---%%%---%%%---%%%---%%%---%%%---%%%---%%%---%%%---%%%---%%%---%%%---%%%---%%%---%%%---%%%---%%%---%%%---%%%---%%%---%%%
	As we hinted before, the fact that the system display all the information at once can be quite overwhelming and difficult to use in practice. That is without considering the different filters that helps narrowing down the information to exactly the one we need to detect mistakes in our grammar.

	\bigskip

	The different filters buttons can be observed on the number 2 highlight on figure \ref{fig:stand_alone_detailed}.

	\begin{itemize}
		\item ``Filter no match'', due to the nature of parser definition borrowed from the PEG formalism, certain parsers can potentially return successfully although they didn't advance the parse in any ways. An example would be the ``repeat0'' parser which return successfully uppon matching 0 or more times the content of its body. Such parser might get in the way of clarity when we analyze how the parse actually matched the input. These parser can be identified by the fact that the position before and after their invocation is the same. For this reason we can filter them out using this button. 
		\item ``Filter backtracked'', due to context sensitivity, some parsers might match the input partially and then backtracked upon failure. Similarily to successful parsers that didn't match the input, depending on the situation, those entries can get in the way of understanding how the input was matched. It is although obviously important to keep those parsers around within the syntax tree to debug our solution. A particular rule we wrote could backtrack unexpectedly, we therefore might want to analyze why it failed to match the input.
		\item ``All nodes'', as its name indicate, this radio button displays all the nodes without filtering them based on their name or type.
		\item ``Named rule'', this radio button filter the tree to leave only the parsers that are the begining of a rule definition. Filtering named parsers is particularly useful as each nodes of the tree now match directly with the rules of the grammar, making it really easy to understand how the parse matched the input file.
		\item ``Filtered'', Finally, this radio button works together with the search field marked as the number 3 on figure \ref{fig:stand_alone_detailed}. One can type rule names or parser type name in the search field to filter out from the tree all the parsers that doesn't match this definition. Switching to this radio button simply filter the tree with whatever's in the search field. Intrinsicly, this radio button doesn't do much except indicating that the display is driven by the search field, which overides the other name based filters. 
	\end{itemize}

	%%%---%%%---%%%---%%%---%%%---%%%---%%%---%%%---%%%---%%%---%%%---%%%---%%%---%%%---%%%---%%%---%%%---%%%---%%%---%%%---%%%---%%%---%%%---%%%---%%%
	\subsection{Jump to code}
	%%%---%%%---%%%---%%%---%%%---%%%---%%%---%%%---%%%---%%%---%%%---%%%---%%%---%%%---%%%---%%%---%%%---%%%---%%%---%%%---%%%---%%%---%%%---%%%---%%%

	A really interesting feature for our plugin is the ability to jump directly in the code for the definition of the grammar rule and/or the input file. Our implementation doesn't allow us to do it in a neat way. Right now, we achieve this feature by selecting one rule from the table or tree view, then going to the ``tool'' menu and selection ``jump to rule definition''. Alternatively, there is a keyboard shortcut to do this : command-alt-A then C.

	The reason for this will be discussed in the chapter \ref{chap:impl}.


	% However, our implementation doesn't allow for this feature to be easily accessible. The problem lies in the way Intellij accepts 

	% Through another menu, we are able open the file and move the caret to highlight the relevant information.


		%PetitParser is a framework for creating parsers, written in Pharo, that makes it easy to dynamically reuse, compose, transform and extend grammars [28]. A parser is created by specifying a set of grammar productions in one or more dedicated classes. When a parser is instantiated the grammar productions are used to create a tree of primitive parsers (e.g., choice, sequence, negation, etc.); this tree is then used to parse the input. Whereas most parser generators instantiate a parser by generating code, PetitParser generates a dynamic graph of objects. Nevertheless, the same issues arise as with conventional parser generators: generic debuggers do not provide debugging operations at the level of the input (e.g., set a breakpoint when a certain part of the input is parsed) and of the grammar (e.g., set a breakpoint when a grammar production is exercised). Generic debuggers also do not display the source code of grammar productions nor do they provide easy access to the input being parsed.

