%%%---%%%---%%%---%%%---%%%---%%%---%%%---%%%---%%%---%%%---%%%---%%%---%%%---%%%---%%%---%%%---%%%---%%%---%%%---%%%---%%%---%%%---%%%---%%%---%%%
%
\chapter{Implementation}
\label{chap:impl}
%
%%%---%%%---%%%---%%%---%%%---%%%---%%%---%%%---%%%---%%%---%%%---%%%---%%%---%%%---%%%---%%%---%%%---%%%---%%%---%%%---%%%---%%%---%%%---%%%---%%%
	This chapter will present the practical implementation of the solution.

	In section \ref{sec:autumn}, we presented the overall structure of the Autumn parsing libraries. 

	The debugging logic has been implemented using hooks attached to the parsers.

%%%---%%%---%%%---%%%---%%%---%%%---%%%---%%%---%%%---%%%---%%%---%%%---%%%---%%%---%%%---%%%---%%%---%%%---%%%---%%%---%%%---%%%---%%%---%%%---%%%
\section{The debuging tool implementation}
%
%%%---%%%---%%%---%%%---%%%---%%%---%%%---%%%---%%%---%%%---%%%---%%%---%%%---%%%---%%%---%%%---%%%---%%%---%%%---%%%---%%%---%%%---%%%---%%%---%%%

%%%---%%%---%%%---%%%---%%%---%%%---%%%---%%%---%%%---%%%---%%%---%%%---%%%---%%%---%%%---%%%---%%%---%%%---%%%---%%%---%%%---%%%---%%%---%%%---%%%
	\subsection{Overview of the structure}
%
%%%---%%%---%%%---%%%---%%%---%%%---%%%---%%%---%%%---%%%---%%%---%%%---%%%---%%%---%%%---%%%---%%%---%%%---%%%---%%%---%%%---%%%---%%%---%%%---%%%

	\begin{itemize}
		\item model
		\begin{itemize}
			\item builder
			\item model compilers
			\begin{itemize}
				\item model compiler
				\item graph model compiler
			\end{itemize}
		\end{itemize}
		\item parsers interface, subparsers
		\item syntax tree STnode
	\end{itemize}

\begin{itemize}
	\item autumn
	\begin{itemize}
		\item parse state
		\begin{itemize}
			\item undoable list
			\item undoable map
		\end{itemize}
		\item grammar
	\end{itemize}
\end{itemize}

%%%---%%%---%%%---%%%---%%%---%%%---%%%---%%%---%%%---%%%---%%%---%%%---%%%---%%%---%%%---%%%---%%%---%%%---%%%---%%%---%%%---%%%---%%%---%%%---%%%
	\subsection{Hooking into the parser implementation}
%
%%%---%%%---%%%---%%%---%%%---%%%---%%%---%%%---%%%---%%%---%%%---%%%---%%%---%%%---%%%---%%%---%%%---%%%---%%%---%%%---%%%---%%%---%%%---%%%---%%%
	The debuging logic presented in the previous chapter has been implemented through the creation of a hook inside the parsers. During a debuging session, each parser invocation will instead call the debug function that will handle the debugging logic.

	\bigskip

	Originally, the parsers were implemented in a strictly functionnal fashion. Autumn defines a parser simply as a boolean function taking some input, optionally modifying the parse state, advancing the input position and finally returning a boolean indicating the success or failure of its execution. 

	\bigskip

	To implement our hook, we changed the parsers from a functional implementation to an object oriented implemention by simply wrapping the parsers function inside objects.
	A general parser interface has been created to serve as an entry point for our hook.

	\bigskip

	We called this new implementation ``naive parsers'' because we suspected that it would cause some performances issues due to constant creation of new objects during the parse, this issue was referred to as megamorphic call sites. Interestingly, after comparing the benchmark, it has been shown that the impact on the execution time wasn't significant. The results of these benchmark is discussed in chapter \ref{chap:bench}

%%%---%%%---%%%---%%%---%%%---%%%---%%%---%%%---%%%---%%%---%%%---%%%---%%%---%%%---%%%---%%%---%%%---%%%---%%%---%%%---%%%---%%%---%%%---%%%---%%%
%
	\subsection{Rewritting the grammar, the notion of model and model Compiler}
%
%%%---%%%---%%%---%%%---%%%---%%%---%%%---%%%---%%%---%%%---%%%---%%%---%%%---%%%---%%%---%%%---%%%---%%%---%%%---%%%---%%%---%%%---%%%---%%%---%%%

	The changes made to the parsers had the consequence that we had to rewrite the grammar to reflect those modifications. As discussed before, writing grammar without proper debugging tools can be a time consuming and error prone business. Moreover, to investigate the suspected performances issues related to the megamorphic call sites, we wanted to make sure that we could recreate this new grammar in such a way that we could effectively compare its performances with the previous implementation.

	\bigskip

	To do so, we created an \textbf{object graph model} representing the grammar. This model would be used to generate actual grammars using a model compiler. The first challenge was to regenerate the exact same java grammar we had before based on the model. The reason for this step was to ensure the correctness of the model, if we can regenerate the grammar from the model, then we can be sure that another grammar generate from the same model will be correct as well.

	The second compiler, baptise ``graph model compiler'' aimed to generate a java grammar based on the object oriented implementation of the parsers. It's name comes from the fact that the grammar generated is represented by a graph of connected objects. Figure \ref{fig:model-grammar} illustrate the relationship between model, model compiler and grammar.

	\bigskip

	Then, both grammar was tested on a consequential java corpus. It appeared that the performances, although slightly affected by the creation of so many objects wasn't impacted in a significant way. The results will be discussed more deeply in the chapter \ref{chap:bench}. 

	\begin{figure}[h]
		\centering
		\includegraphics[width=1\textwidth] {ressources/model-grammar}
		\caption{Chart illustrating model-grammar generation} 
		\label{fig:model-grammar}
	\end{figure}

	% \clearpage
	\bigskip

	The other motivation we had to create this framework was to simplify the writting of grammar. Autumn grammars are indeed constrained by the inlining notation of the Kotlin language that introduce a less desirable verbosity. The notation for rule definition can be greatly simplified by using this framework as one can see in the following code sample. 

	\begin{tcolorbox}
	\begin{lstlisting}
val method_ref_suffix = Build(2,
   syntax = iden,
   effect = {MaybeBoundMethodReference(it(0), it(1), 
   				it(2))}).g.d("method_ref_suffix")
	\end{lstlisting}
    \end{tcolorbox}
    \begin{tcolorbox}
	\begin{lstlisting}
val method_ref_suffix
   = iden
   .build (2, "MaybeBoundMethodReference(it(0), it(1), it(2))")

	\end{lstlisting}
    \end{tcolorbox}

	% \begin{tcolorbox}
	% \begin{verbatim}
	% val method_ref_suffix = Build(2,
 %        syntax = iden,
 %        effect = {MaybeBoundMethodReference(it(0), it(1), it(2))})
 %        		.g.d("method_ref_suffix")

 %    \end{verbatim}
 %    \end{tcolorbox}
 %    \begin{tcolorbox}
 %    \begin{verbatim}
 %    val method_ref_suffix
 %        = iden
 %        .build (2, "MaybeBoundMethodReference(it(0), it(1), it(2))")

 %    \end{verbatim}
    % \end{tcolorbox}
% 

% \begin{itemize}
% 	\item schemas qui explique la structure model/grammar.
% \end{itemize}

%%%---%%%---%%%---%%%---%%%---%%%---%%%---%%%---%%%---%%%---%%%---%%%---%%%---%%%---%%%---%%%---%%%---%%%---%%%---%%%---%%%---%%%---%%%---%%%---%%%
%
		\subsection{Syntax Tree generation}
%
%%%---%%%---%%%---%%%---%%%---%%%---%%%---%%%---%%%---%%%---%%%---%%%---%%%---%%%---%%%---%%%---%%%---%%%---%%%---%%%---%%%---%%%---%%%---%%%---%%%
	
	The AST autumn is building during the parse is abstracting some of the nodes together to reduce the size of the tree and make it more readable from a human stand-point. A consequence of this is that we don't have a direct mapping between tree nodes and grammar rules.

	\bigskip

	This is a critical part of the debuging effort. The syntax tree is build by the debuging logic, separately from autumn AST and is only build during in debugging mode.
 	For debugging purposes, backtracking nodes are kept within the tree as well. Backtracking nodes are of course marked as backtracked

 	\bigskip

	The implementation of the syntax tree is contain in the STNode class. It is the structure that holds all the informations needed to debug the grammar.

%%%---%%%---%%%---%%%---%%%---%%%---%%%---%%%---%%%---%%%---%%%---%%%---%%%---%%%---%%%---%%%---%%%---%%%---%%%---%%%---%%%---%%%---%%%---%%%---%%%

%%%---%%%---%%%---%%%---%%%---%%%---%%%---%%%---%%%---%%%---%%%---%%%---%%%---%%%---%%%---%%%---%%%---%%%---%%%---%%%---%%%---%%%---%%%---%%%---%%%
%
		\subsection{Debug logic and syntax nodes}
%
%%%---%%%---%%%---%%%---%%%---%%%---%%%---%%%---%%%---%%%---%%%---%%%---%%%---%%%---%%%---%%%---%%%---%%%---%%%---%%%---%%%---%%%---%%%---%%%---%%%

	When called the debuging logic will start building up the syntaxt tree as the different parsers get invoked. For each invoked parser, a syntax node is created graft onto the tree. The final syntax tree counts one node per parser invoked.

	One might wonder what happens when a particular parser fail and backtracked over. Nothing happens, we keep this node right inside the tree, because for debugging purposes, it can be interesting to see which node backtracked as a rule could fail to match an input while its intended purpose was to match it.

	\bigskip

	The syntax tree itself is represented by a chained structure called syntax node, or STNode. Each parser call is represented by an STNode in the tree. It holds information about the parser:

	\begin{itemize}
		\item each nodes knows information about the parser it is linked to
		\item its type
		\item if it is the definition of a rule
		\item the position in the input
		\item the size of the state log to be able to regenerate the state
		\item its parent and children
	\end{itemize}

	In the context sensitivity setting, it might be important to be able to display information related to the global parse state. The way autumn recall context is to register it within a mutable parse state. This parse state essentially works as a log whose entries represent mutations applied to this state. Each entry stores the mutation as well as a way to revert it, and later on reapply it. Therefore, it is possible by reccording the size of the log to actually regenerate the parse state corresponding to a specific parser call.

	\bigskip

	This state represent the context shared between different parsers, as such, when one parser backtracked, the mutations it applied to the parse state are reverted and removed from the log. Therefore, to be able to regenerate the parse state for backtracked parsers, we need to adapt our implementation a bit.

	\bigskip

	The debug logic is contained in the \textit{debugNode} class. Any number of additional functionality can be added by creating new hooks to the implementation of the debug function


% Usually in grammar parsing, the framework usually generate a syntax tree and let the user generate an AST derived from it. One of Autumn's specificity was that it didnt produce a Syntax tree and directly produced a AST. The problem with this approach on the debugger side is that we want to be able to see exactly which rule matched with the input and to do se we need the nodes from the tree to be in a direct mapping with the rules of the grammar.

	%%%---%%%---%%%---%%%---%%%---%%%---%%%---%%%---%%%---%%%---%%%---%%%---%%%---%%%---%%%---%%%---%%%---%%%---%%%---%%%---%%%---%%%---%%%---%%%---%%%
	\section{Plugin Implementation}
	%%%---%%%---%%%---%%%---%%%---%%%---%%%---%%%---%%%---%%%---%%%---%%%---%%%---%%%---%%%---%%%---%%%---%%%---%%%---%%%---%%%---%%%---%%%---%%%---%%%
	Intellij IDE's interace uses java SWING. Considering that the entire project has been code in Kotlin, we were reluctant to use this out of style implementation of GUI.

	\paragraph{}

 	Because we needed to our implementation to interoperate with the SWING components of the IDE, we considered javaFX \cite{oracle_javafx} which is an evolution of SWING by Oracle, but like its predecessor it's written in java and still very verbose. In the end, we were seduced by the tornadoFX \cite{tornadofx} framework, coded in kotlin, it is build on top of of javaFX, assuring seamless interoperability with SWING as well as providing the powerful syntax of Kotlin.

		%%%---%%%---%%%---%%%---%%%---%%%---%%%---%%%---%%%---%%%---%%%---%%%---%%%---%%%---%%%---%%%---%%%---%%%---%%%---%%%---%%%---%%%---%%%---%%%---%%%
		\subsubsection{TornadoFX framework - reasons why we used tornadofx}
		%%%---%%%---%%%---%%%---%%%---%%%---%%%---%%%---%%%---%%%---%%%---%%%---%%%---%%%---%%%---%%%---%%%---%%%---%%%---%%%---%%%---%%%---%%%---%%%---%%%
		Present the strenght of tornadoFX vs standard SWING. Highlight the fact that it is written in kotlin and therefore can harness the strenght of the language and at the same time, since its based on javaFX it guaranteed its interoperability with SWING and therefore blend seemlessly with the IDE.

		Present some example of how it can dramatically reduce the effort needed to build up a GUI compared to SWING.
		% User interfaces are becoming increasingly critical to the success of consumer and business applications. With the rise of consumer mobile apps and web applications, business users are increasingly holding enterprise applications to a higher standard of quality. They want rich, feature-packed user interfaces that provide immediate insight and navigate complex screens intuitively. More importantly, they want the application to adapt quickly to business changes on a frequent basis. For the developer, this means the application must not only be maintainable but also evolvable. TornadoFX seeks to assist all these objectives and greatly streamline the coding of JavaFX UI's.
		% While much of the enterprise IT world is pushing HTML5 and cloud-based applications, many businesses are still using desktop UI frameworks like JavaFX. While it does not distribute to large audiences as easily as web applications, JavaFX works well for "in-house" business applications. Its high-performance with large datasets (and the fact it is native Java) make it a practical choice for applications used behind the corporate firewall.
		% JavaFX, like many UI frameworks, can quickly become verbose and difficult to maintain. Fortunately, the release of Kotlin has created an opportunity to rethink how JavaFX applications are built.

		% In February 2016, JetBrains released Kotlin, a new JVM language that emphasizes pragmatism over convention. Kotlin works at a higher level of abstraction and provides practical language features not available in Java. One of the more important features of Kotlin is its 100% interoperability with existing Java libraries and codebases, including JavaFX.
		% While JavaFX can be used with Kotlin in the same manner as Java, some believed Kotlin had language features that could streamline and simplify JavaFX development. Well before Kotlin's beta, Eugen Kiss prototyped JavaFX "builders" with KotlinFX. In January 2016, Edvin Syse rebooted the initiative and released TornadoFX.
		% TornadoFX seeks to greatly minimize the amount of code needed to build JavaFX applications. It not only includes type-safe builders to quickly lay out controls and user interfaces, but also features dependency injection, delegated properties, control extension functions, and other practical features enabled by Kotlin. TornadoFX is a fine showcase of how Kotlin can simplify codebases, and it tackles the verbosity of UI code with elegance and simplicity. It can work in conjunction with other popular JavaFX libraries such as ControlsFX and JFXtras. It works especially well with reactive frameworks such as ReactFX as well as RxJava and friends (including RxJavaFX, RxKotlin, and RxKotlinFX).

		% Motivational example in the intro guide

		%%%---%%%---%%%---%%%---%%%---%%%---%%%---%%%---%%%---%%%---%%%---%%%---%%%---%%%---%%%---%%%---%%%---%%%---%%%---%%%---%%%---%%%---%%%---%%%---%%%
		\subsection{Event Handler \& Message passing}
		%%%---%%%---%%%---%%%---%%%---%%%---%%%---%%%---%%%---%%%---%%%---%%%---%%%---%%%---%%%---%%%---%%%---%%%---%%%---%%%---%%%---%%%---%%%---%%%---%%%
		implementation details of the GUI that are maybe not so relevant as we discussed ... I suppose Im gonna remove this section.



% \paragraph{side note}
% An interesting product that is in development is JPro, a web-based JavaFX container that uses no plugins. It can work with TornadoFX and JavaFX, but is still in closed beta at the time of writing. You can follow the project and wait for its availability here: https://jpro.io/

	%%%---%%%---%%%---%%%---%%%---%%%---%%%---%%%---%%%---%%%---%%%---%%%---%%%---%%%---%%%---%%%---%%%---%%%---%%%---%%%---%%%---%%%---%%%---%%%---%%%
	\subsection{Limitations}

	%%%---%%%---%%%---%%%---%%%---%%%---%%%---%%%---%%%---%%%---%%%---%%%---%%%---%%%---%%%---%%%---%%%---%%%---%%%---%%%---%%%---%%%---%%%---%%%---%%%

		\subsection{execution thread}
		\begin{itemize}
			\item intellij doesnt allow certain instructions to be called from another thread than the one expects.
			\item tornadoFX works in its own thread, so some functionallity are not possible as it is
		\end{itemize}